\documentclass[12pt]{report}

\usepackage{researchPaper}

\graphicspath{{images/}}

\bibliography{QEPaper.bib}

\begin{document}

%Arguments #1=title #2=authors
\newcommand{\titlesection}[2]
{
    \title{#1}
    \lhead{Benjamin Ojeda}
    \rhead{#1}
    \author{#2}
    \date{2016-03-28}
    \maketitle
}

\titlesection{Quantitative Easing Reading Response}{Benjamin Ojeda}

%Arguments #1=image #2=location #3=width #4=caption #5=citation
\newcommand{\wrapfig}[5]
{
    \begin{wrapfigure}{#2}{#3\textwidth}
        \label{fig:#1}
        \vspace{-22pt}
        \begin{tcolorbox}[left=1mm,right=1mm,top=1mm,bottom=1mm,colback=blue!6!white,boxrule=0mm,colframe=blue!75!black,arc=0pt,outer arc=0pt]
            \centering
            \includegraphics[width=\textwidth]{#1}
            \vspace{-30pt}
            \caption{#4\protect\footnotemark}
            %\vspace{-18pt}
        \end{tcolorbox}
        \vspace{-22pt}
    \end{wrapfigure}
    \footnotetext{#5}
}

%Arguments #1=image #2=width #3=caption #4=citation
\newcommand{\fig}[4]
{
    \begin{figure}[!ht]
        \centering
        \includegraphics[scale=#2]{#1}
        \caption{#3\footnotemark}
        \label{fig:#1}
    \end{figure}
    \footnotetext{#4}
}

The financial crisis of 2007-20088 pushed the traditional tools of monetary policy to their limits, forcing central banks across the world to adopt unconventional policies in their efforts to cope with the downturn.
Prior to the crisis, the aim of monetary policy was low, stable inflation. This policy of inflation targeting was relatively uncontroversial; central banks manipulated the interest rate on funds provided to banks.
The impact of the changes on the broader market was straightforward and predictable and could be reliably quantified.\autocite[271]{joyce2012quantitative}
Most central banks adjusted these interest rates in response to a range of indicators, generally sticking to the principles laid out by the "Taylor Rule."\autocite[271]{joyce2012quantitative}
The Taylor rule is a principle, first proposed by John B. Taylor in 1993, that aims to approximate the target federal funds rate.\autocite[232]{woodford2001taylor}
The equation takes the form
\begin{center}
$i_t = 0.04 + 1.5(pi_t -0.02) + 0.5(y_t - \bar{y}_t)$
\end{center}
where $i_t$ represents the target federal funds rate, $\pi_t$ is the inflation rate measured by the GDP deflator, $y_t$ is the log of real GDP, and $\bar{y}_t$ is the potential output.\autocite[232]{woodford2001taylor}
To elaborate briefly upon this equation, since the difference of two logs is the same as the logarithm of their division the expression $(y_t - \bar{y}_t)$ is equivalent to $\mbox{log}(\frac{GDP}{Capacity})$ which will be negative since GDP is always a fraction of capacity.
In brief, the Fed will target a lower federal funds rate in times where the economy falls well bellow capacity.
This rule has proven very reliable in predicting the Fed's target rate over the years,\autocite[232]{woodford2001taylor} but during the crisis of 2007-2008 the Taylor rule got pushed beyond its breaking point as the economy fell to the point were the equation actually prescribed \emph{negative} interest rates.
However, market interest rates cannot be pushed beyond zero because at that point even cash offers better returns than interest-bearing assets.\autocite[272]{joyce2012quantitative}
This has forced central banks to reach for more unconventional tools in order to stimulate the economy and provide liquidity.
Chief amongst these unconventional tools has been quantitative easing, or QE, which has risen to prominence in the past few years as central banks across the world have adopted it to allow them to continue to exercise monetary policy in an environment of near-zero percent overnight interest rates.

\wrapfig{Fed_BalanceSheet}{r}{.4}{Federal Reserve Balance Sheet vs Policy Rate}{\autocite[273]{joyce2012quantitative}}

The basic assumption of traditional monetary policy is that by altering interest rates a central bank can change asset prices so as to alter banks' willingness to lend, and firms' and individuals' desire to invest and consume.\autocite[51]{fawley2013four}
The principle behind QE is that even in a situation where the overnight rate -- which is effectively risk free -- is bound at zero, the central bank can still try to exercise monetary policy by manipulating interest rate spreads.\autocite[466]{blinder2010quantitative}
Both longer-term and higher-risk debt command premiums, so at least in principle as long as the central bank can reduce these premiums, it could still increase aggregate demand even in a situation where the lower bound of the yield curve is at zero.\autocite[466]{blinder2010quantitative}
There are two ways for a central bank to fund these activities, either it could sell off its short-term securities, which would change the composition of its balance sheet, or it could create new currency which would increase the size of its balance sheet.\autocite[467]{blinder2010quantitative}

Quantitative easing or QE was first tested as a policy in Japan in the early 2000s as the nation dealt with the aftermath of a devastating real estate bubble.\autocite[274]{joyce2012quantitative}
The name \emph{quantitative easing} was meant to indicate a shift in policy towards targeting quantity variables rather than just interest rates as the Bank of Japan began to purchase government securities from banks in order to boost their cash reserves.\autocite[274]{joyce2012quantitative}
The Bank of Japan's efforts to restart Japan's economy through QE proved to be deeply disappointing and very little progress had been made by 2006 when the BoJ finally brought its QE program to a close.\autocite[55]{fawley2013four}
This first try at QE was mostly inconclusive with regards to its efficacy as a tool of monetary policy, and some have argued that the BoJ did a poor job managing market expectations which may have critically undermined their QE policy.\autocite[55]{fawley2013four}
One important aspect of quantitative easing is that the greatest effects of the policy can only be felt if investors see a credible commitment on the part of the central bank to keep interest rates low, even if the economy recovers and the Taylor Rule prescribes increased interest rates.\autocite[4]{krishnamurthy2011effects}
The central bank can signal such a commitment through public statements along with purchasing large quantities of long-term assets during QE.\autocite[4]{krishnamurthy2011effects}
By the time the financial crisis of 2007 struck, the BoJ was still maintaining rock bottom interest rates, having failed to restore their economy to normal functioning order after more than a decade of economic tumult.\autocite[56]{fawley2013four}

The Fed's first foray into QE started in early 2008 when it began selling its short term Treasuries in exchange for less-liquid assets with the aim of reducing the premium placed on liquidity, especially given the panicked state of the overall financial markets.\autocite[467]{blinder2010quantitative}
As the crisis worsened following the collapse of Lehman Brothers, Fed Chairman Ben Bernanke initiated what came to known as the first big round of QE. He initially chose to refer to the policy as ``credit easing'' probably to avoid association with the BoJ's prior failures.\autocite[465]{blinder2010quantitative}
Technically CE is a special case of QE, since the Fed sought to to lower long-term interest rates it was focused only on a particular market/interest rate which is CE, as opposed to QE which can refer to any policy that raises the liabilities on central bank balance sheets.\autocite[55]{fawley2013four}
Regardless of this distinction, the first round of QE known as QE1 -- saw the Federal Reserve expand its balance sheet from \$907 billion on September 3rd to \$2,214 billion on November 12th of 2008 just over two months later.\autocite[468]{blinder2010quantitative}
The extent of these initial rounds of QE can be seen clearly in figure \ref{fig:CompositionOfFedBalanceSheet}.
The summer of 2010 was marked by fears of a possible deflationary cycle, similar to what beset Japan during its lost decade.\autocite[1]{ricketts2014rise}
These concerns prompted a second round of QE under which the Fed further expanded its balance sheet, purchasing \$600 billion in long-term securities between November 2010 and June 2011.\autocite[1]{ricketts2014rise}
In spite of these efforts the economy remained weak and the European sovereign debt crisis brought fresh uncertainty to global markets.
The Fed began a program called Operation Twist in which it sold \$667 billion worth of short-term treasury securities in favor of an equal quantity of long-term treasury securities.\autocite[1]{ricketts2014rise}
The fed began yet another round of QE in 2013\autocite[1]{ricketts2014rise} before finally bringing the program to a close in 2014 by which time the Fed's balance sheets had reached a historic \$4.5 trillion worth of assets.\autocite{nytFed}
The United States is not the only nation that has been forced to turn to QE, the European Central Bank was forced to begin its own QE program in early 2015.\autocite{economistQEGermany}

\iffalse
There is considerable concern that the policy will not be as effective in Europe as it has proven in the United States.\autocite{economistQEGermany}
Prior to Quantitative Easing, Europe had actually begun to enter deflation indicating that popular concern that the policy would lead to hyperinflation are largely irrational.\autocite{economistQEGermany}
\autocite{economistQEGermany}
\fi

On the whole, quantitative easing has proven to be a relatively successful policy.
Although the Fed balance sheet is immense, it remains a managable fraction of nominal GDP in line with levels seen during other historic periods as shown in figure \ref{fig:HistOfFedBalance}.
Eventually the Fed will be able to unload excess reserves once the economy recovers.\autocite[4]{ricketts2014rise}
Although growth remains anemic across most of the developed world, and the recovery from the great recession has proved disappointing, it is unreasonable to attribute these failures to monetary policy.
Convention holds that the federal funds rate is a far more effective and reliable tool than QE.\autocite[465]{blinder2010quantitative}
Ultimately, the global economy is facing many deep structural problems which cannot be resolved through stopgap monetary policies such as quantitative easing.


\begin{appendices}

\fig{CompositionOfFedBalanceSheet}{.5}{Composition of the Fed's Balance Sheet}{\autocite[469]{blinder2010quantitative}}

\fig{HistOfFedBalance}{.5}{Fed balance sheet as a share of nominal GDP}{\autocite[2]{ricketts2014rise}}

\end{appendices}

\nocite{*}
\printbibliography

\end{document}
