\documentclass[12pt]{report}

\usepackage{researchPaper}

\graphicspath{{images/}}

\bibliography{QEPaper.bib}

\begin{document}

%Arguments #1=title #2=authors
\newcommand{\titlesection}[2]
{
    \title{#1}
    \lhead{Benjamin Ojeda}
    \rhead{#1}
    \author{#2}
    \date{2016-03-28}
    \maketitle
}

\titlesection{Quantitative Easing Reading Response}{Benjamin Ojeda}

%Arguments #1=image #2=location #3=width #4=caption #5=citation
\newcommand{\wrapfig}[5]
{
    \begin{wrapfigure}{#2}{#3\textwidth}
        \label{fig:#1}
        \vspace{-22pt}
        \begin{tcolorbox}[left=1mm,right=1mm,top=1mm,bottom=1mm,colback=blue!6!white,boxrule=0mm,colframe=blue!75!black,arc=0pt,outer arc=0pt]
            \centering
            \includegraphics[width=\textwidth]{#1}
            \vspace{-30pt}
            \caption{#4\protect\footnotemark}
            %\vspace{-18pt}
        \end{tcolorbox}
        \vspace{-22pt}
    \end{wrapfigure}
    \footnotetext{#5}
}

%Arguments #1=image #2=width #3=caption #4=citation
\newcommand{\fig}[4]
{
    \begin{figure}[!ht]
        \centering
        \includegraphics[scale=#2]{#1}
        \caption{#3\footnotemark}
        \label{fig:#1}
    \end{figure}
    \footnotetext{#4}
}

The financial crisis of 2007-8 pushed the traditional tools of monetary policy to their limits, forcing central banks across the world to adopt unconventional policies in their efforts to cope with the downturn.
Prior the crisis the aim of monetary policy was low stable inflation. This policy of inflation targeting was relatively uncontroversial, central banks manipulated the interest rate on funds provided to banks, the impact of the changes on the broader was straightforward and predictable and could be reliably quantified.\autocite[271]{joyce2012quantitative}
Most central banks adjusted these interest rates in response to a broad range of indicators, generally sticking to the principles laid out by the "Taylor Rules".\autocite[271]{joyce2012quantitative}
The Taylor rule is a principle, first proposed by John B. Taylor in 1993, that aims to approximate the target federal funds rate.\autocite[232]{woodford2001taylor}
The equation takes the form
\begin{center}
$i_t = 0.04 + 1.5(pi_t -0.02) + 0.5(y_t - \bar{y}_t)$
\end{center}
where $i_t$ represents the target federal funds rate, $\pi_t$ is the inflation rate measured by the GDP deflator, $y_t$ is the log of real GDP, and $\bar{y}_t$ is the potential output.\autocite[232]{woodford2001taylor}
To elaborate briefly upon this equation, since the difference of two logs is the same as the logarithm of their division $(y_t - \bar{y}_t)$ is equivalent to $\mbox{log}(\frac{GDP}{Capacity})$ which will be negative since GDP is always less than potential Output.
In brief, the Fed will target a lower federal funds rate in times where the economy falls well bellow capacity.
This rule has proven very reliable in predicting the Fed's target rate over the years,\autocite[232]{woodford2001taylor} but during the crisis of 2007-2008 the Taylor rule got pushed beyond its breaking point as the economy fell to the point were the equation actually prescribed \emph{negative} interest rates.
However, market interest rates cannot be pushed beyond zero because at that point even cash offers better returns than interest-bearing assets.\autocite[272]{joyce2012quantitative}
This has forced central banks to reach for more unconventional tools in order to stimulate the economy and provide liquidity.
Chief amongst these unconventional tools has been Quantitative Easing, or QE, which has risen to prominence in the past few years as central banks across the world have adopted it to allow them to continue to exercise monetary policy in an environment of near-zero percent overnight interest rates.

\wrapfig{Fed_BalanceSheet}{r}{.4}{Federal Reserve Balance Sheet vs Policy Rate}{\autocite[273]{joyce2012quantitative}}

Quantitative Easing or QE was first tested as a policy in Japan in the early 2000s as the nation dealt with the aftermath of a devastating real estate bubble.\autocite[274]{joyce2012quantitative}
The name \emph{Quantitative Easing} was meant to indicate a shift in policy towards targetting quantity variables rather than just interest rates as the Bank of Japan began to purchase government securities from banks in order to boost their cash reserves.\autocite[274]{joyce2012quantitative}
The Bank of Japans efforts to restart Japan's economy through QE proved to be deeply dissapointing and very little progress had been made by 2006 when the BoJ finally brought its QE program to a close.\autocite[55]{fawley2013four}
These first try at QE was mostly inconclusive with regards to the efficacy of QE as a tool of monetary policy, and some have argued that the BoJ did a poor job managing market expectations which may have critically undermined their QE policy.\autocite[55]{fawley2013four}
One important aspect of Quantitative easing is that the greatest effects of the policy can only be felt if investors see a credible commitment on the part of the central bank to keep interest rates low, even if the economy recovers and the Taylor rule prescribes increased interest rates.\autocite[4]{krishnamurthy2011effects}
The central bank can signal such a commitment through public statements along with purchasing large quantities of long-term assets during QE.\autocite[4]{krishnamurthy2011effects}
Regardless by the time the fincial crisis of 2007 struck, the BoJ was still at rock bottom interest rates, having failed to restore their economy to normal functioning order after more than a decade of economic tummult.\autocite[56]{fawley2013four}

When Fed Chairman Ben Bernanke initiated the first round of QE he chose to refer to the policy as "credit easing" in a failed attempt to avoid similar actions taken by the Bank of Japan.\autocite[465]{blinder2010quantitative}
Technically CE is a special case of QE, since the Fed sought to to lower long-term interest rates it was focused only on a particular market/interest rate which is CE, as opposed to QE which can refer to any policy that raises the liabilities on central bank balance sheets.\autocite[55]{fawley2013four}
Regardless of this distinction, the policies that the Fed would pursue over the years following the great recession came to be known as Quantiative Easing.
The Fed's first foray into QE started during early 2008 when the Fed started selling its short term Treasuries in exchange for less-liquid assets with the aim of reducing the premium placed on liquidity, especially given the panicked state of the overall financial markets.\autocite[467]{blinder2010quantitative}
By November of 2008 the effective Federal funds rate had been pushed to zero, and the Fed began to turn towards more QE as the Federal Reserve expanded its balance sheet from \$907 on September 3rd, 2008 to \$2,214 billion on November 12th.\autocite[468]{blinder2010quantitative}

\autocite[]{ricketts2014rise}



The basic assumption of traditional monetary policy is that by altering interest rates a central bank can change asset prices so as to alter banks willingness to lend, and firms and individuals to invest and consume.\autocite[51]{fawley2013four}
The principle behind QE is that even in a situation where the overnight rate - which is effectively risk less - is bound at zero, the central bank can still try to exercise monetary policy by manipulating interest rate spreads.\autocite[466]{blinder2010quantitative}
Both longer-term and higher-risk debt command premiums, so at least in principle as long as the central bank can reduce these premiums, it could still increase aggregate demand even in a situation where the lower is at zero.\autocite[466]{blinder2010quantitative}

The European Central Bank began its own program of Quantitative Easing early in 2015.\autocite{economistQEGermany}
There is considerable concern that the policy will not be as effective in Europe as it has proven in the United States.\autocite{economistQEGermany}
Prior to Quantitative Easing, Europe had actually begun to enter deflation indicating that popular concern that the policy would lead to hyperinflation are largely irrational.\autocite{economistQEGermany}
\autocite{economistQEGermany}


On a whole, Quantitative Easing has proven to be a relatively successful policy.
Although growth remains anemic across most of the developed world, and the recovery from the great recession has proven dissapointing, it is unreasonable to attribute these failures to monetary policy.
Convention holds that the or federal funds rate is a far more effective and reliable tool than QE.\autocite[465]{blinder2010quantitative}
The global economy is facing many deep structural problems which would have likely proven problematic even without the crisis precipated by the bursting US housing bubble in 2007.

It is difficult to know how long central banks will continue to hold interest rates at zero precent

\begin{appendices}

\fig{CompositionOfFedBalanceSheet}{.4}{Composition of the Fed's Balance Sheet}{\autocite[469]{blinder2010quantitative}}

\end{appendices}

\nocite{*}
\printbibliography

\end{document}
